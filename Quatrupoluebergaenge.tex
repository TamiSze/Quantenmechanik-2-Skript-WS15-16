\subsection{Quatrupolübergänge etc.}
	Matrixelement:
		\begin{align*}
			\braket{f | e^{-i \vec{k} \vec{r}} \vec{\epsilon} \cdot \vec{p} | i} \\
			e^{-i \vec{k} \vec{r}} &= 
			\sum_{j = 0}^{\infty} \frac{(-i)^j}{j!} 
			\underbrace{(\vec{k} \vec{r})^j}_{\mathclap{\mathscr{O} (Z \alpha)^j}} \\
			\text{wobei gilt:~} r &\approx Z a_B ,&
			|\vec{k}| &= \frac{\omega}{c} = \frac{\hbar \omega}{\hbar c} 
			\sim \frac{Z^2 \alpha \hbar c}{2 a_B \hbar c} 
			& &\rightarrow |\vec{k}| r \sim \frac{Z \alpha}{2} \\
			\text{kann nicht sein, einsetzen!}\\
			\text{Wir nehmen lieber:~} r &\approx \frac{a_B}{Z}
		\end{align*} 
		\begin{align*}
			\Rightarrow e^{-i \vec{k} \vec{r}} &=
			\underbrace{1}_{\mathclap{\text{Dipoloperator}}}
			- \overbrace{i \vec{k} \vec{r}}^{\mathclap{\mathscr{O} (Z \alpha)}}
			+ \underbrace{\ldots}_{\mathclap{\mathscr{O} (Z\alpha)^2}}
		\end{align*}
	Was ist mit $(\vec{k} \cdot \vec{r})(\vec{\epsilon} \cdot \vec{p})$?
		\begin{align*}
			(\vec{k} \cdot \vec{r})(\vec{\epsilon} \cdot \vec{p}) &=
			\frac{1}{2} \left[
				(\vec{k} \cdot \vec{r})(\vec{\epsilon} \cdot \vec{p})
				+ (\vec{\epsilon} \cdot \vec{r})(\vec{k} \cdot \vec{p})
				+ (\vec{k} \cdot \vec{r})(\vec{\epsilon} \cdot \vec{p})
				- (\vec{\epsilon} \cdot \vec{r})(\vec{k} \cdot \vec{p})
			\right] \\
			&= \underbrace{\frac{1}{2} \left[
				(\vec{k} \cdot \vec{r})(\vec{\epsilon} \cdot \vec{p})
				+ (\vec{\epsilon} \cdot \vec{r})(\vec{k} \cdot \vec{p})
			\right]}_{\mathclap{\text{elektrischer Quadrupol}}}
			+ \underbrace{\frac{1}{2} 
			(\vec{k} \times \vec{\epsilon})(\vec{r} \times \vec{p})}_{\mathclap{\text{magnetischer Dipol}}}
		\end{align*}
	Nützliche Relation: 
		\begin{equation*}
			\vec{p} = \frac{im}{\hbar} [H^0, \vec{r}]
		\end{equation*}
	Auswahlregeln:
	
	\begin{tabular}{l l} 
		$3d \rightarrow 1s$ & verboten als Dipolübergang, aber erlaubt als elektrischer Quadrupolübergang \\
		$3d \rightarrow 2p \rightarrow 1s$ & ist elektrischer Dipolübergang und hat höhere Rate
	\end{tabular}