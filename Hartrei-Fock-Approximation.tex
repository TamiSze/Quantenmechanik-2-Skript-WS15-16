\subsection{Hartrei-Fock-Approximation}
	\begin{align*}
		H &= \sum_i H^{(i)} + \sum_{i<j} V_{ij}
	\end{align*}
$H^{(i)}$ ist Hamiltonopertor für nicht mit anderen (ununterscheidbaren) Fermionen wechselwirkendes Fermion $i$.
	\begin{align*}
		H \psi_0 &= E_0 \psi_0 
		&\text{mit } E_0 &= \underset{\braket{\psi | \psi} = 1}{\mathrm{min}} \braket{\psi | H | \psi}
	\end{align*}
Grundzustandsenergie

Approximation:
	\begin{align*}
		\psi (1, \ldots, N) 
		&= \frac{1}{\sqrt{N!}} \det 
		\begin{pmatrix}
		\phi_{1}(1) & \cdots & \phi_{1}(N) \\
		\vdots & \ddots & \vdots\\
		\phi_{N}(1) & \cdots & \phi_{N}(N)
		\end{pmatrix} 
	\end{align*}
In der Matrix jeweils in der Klammer stehen Position, Spin etc. und unten sind die Quantenzahlen.
	\begin{align*}
		\phi_i (k) &=
		\underbrace{\phi_i (\vec{r}_k)}_{\mathclap{\text{Bahnwellenfunktion}}}
		\overbrace{\chi_i (\sigma_k)}^{\mathclap{\text{Spinwellenfunktion}}} \\
		\text{Nebenbedingung } &\braket{\phi_i | \phi_j} = \delta_{ij}
	\end{align*}
Dies schränkt nicht die Algemeinheit ein.

Normierung:
	\begin{align*}
		1 &= \braket{\psi | \psi} = \prod_i \braket{\phi_i | \phi_i} 
		& \phi_i &\mapsto \lambda_i \phi_i \\
		1 &= \braket{\psi | \psi} 
		&\mapsto \prod_i |\lambda_i|^2 \braket{\phi_i | \phi_i} &= \prod_i |\lambda_i|^2
		&\Rightarrow 1 &= \prod_i |\lambda_i|^2 = 1 
	\end{align*}
Andere Normierung ist im Prinzip möglich, so lange $\prod_i |\lambda_i|^2 = 1$.

Addition einer Linearkombination $\sum \limits_{j \neq i} c_j \phi_j$ zu $\phi_i$ ändert die Determinante nicht.
	\begin{align*}
		\braket{\psi | H | \psi} &=
		\sum_{i=1}^{N} \braket{\phi_i | H^{(i)} | \phi_i}
		+ \sum_{i<j} \braket{\phi_i | \braket{\phi_j | V_{ij} | \phi_j} | \phi_i}
		- \sum_{i<j} \braket{\phi_i | \braket{\phi_j | V_{ij} | \phi_i} | \phi_j} \\
		&= \sum_i \int \diff^3 r_i \phi_i^* (\vec{r}_i) H^{(i)} \phi_i (\vec{r}_i) 
		+ \sum_{i<j} \int \diff^3 r_i \diff^3 r_j |\phi_i (\vec{r}_i)|^2 |\phi_j (\vec{r}_j)|^2 V_{ij} -\\
		&- \sum_{i<j} \int \diff^3 r_i \diff^3 r_j \phi_i^* (\vec{r}_i) \phi_j^* (\vec{r}_j) 
		 \phi_i (\vec{r}_i) \phi_j (\vec{r}_j) V_{ij}(\vec{r}_i, \vec{r}_j) 
		 \delta_{S_z^{(i)} S_z^{(j)}}
	\end{align*}
Nebenbedingung
	\begin{align*}
		\delta \left[ \braket{\psi | H | \psi} - \sum_{i,j} (\braket{\phi_i | \phi_j} - \delta_{ij}) \lambda_{ij}
		\right] &= 0 \\
		\mathrm{Im} [\ldots] &= 
		\frac{1}{2i} \left(\delta[\ldots] - \delta[\ldots]^*\right) \\
		&= \frac{1}{2i} \sum_{i,j} \left(\lambda_{ji} - \lambda_{ij}^*\right)
		\delta \braket{\phi_i | \phi_i} = 0
	\end{align*}
$\Rightarrow \lambda_{ij} = \lambda_{ji}^*$ und $(\lambda_{ii})$ ist Hermitesche Matrix. \marginpar{warum lambda ii?}
	\begin{align*}
		\delta \int \diff^3 r_i \phi_i^* (\vec{r}_i) H^{(i)} \phi_i (\vec{r}_i)
		&= \int \diff^3 r_i
		\left(
			\delta \phi_i^* (\vec{r}_i) H^{(i)} \phi_i (\vec{r}_i) 
			+ \phi_i^* (\vec{r}_i) H^{(i)} \delta \phi_i (\vec{r}_i) 
		\right) \\
		\delta [\ldots] &= \sum_i \int \diff^3 r_i \delta \phi_i^* (\vec{r}_i)
		\left[
			M \phi_i(\vec{r}_i) - \sum_j \lambda_{ij} \phi_j (\vec{r}_i)
		\right] + \text{c.c.}
	\end{align*}
Mit
	\begin{align*}
		M \phi_i (\vec{r}_i) &=
		H^{(i)} \phi_i (\vec{r}_i) + \sum_{j \neq i} \int \diff^3 r_j |\phi_j (\vec{r}_i)|^2 V_{ij} \phi_i(\vec{r}_i) \\
		&- \sum_{j \neq i} \int \diff^3 r_j \phi_j^* (\vec{r}_j) \phi_j (\vec{r}_i) V_{ij}
		\phi_i (\vec{r}_j) \delta_{S_z^{(i)} S_z^{(j)}} \\
		&= \left[H^{(i)} + C_i - A_i\right] \phi_i(\vec{r}_i)
	\end{align*}
Hier $C_i$ ist Coulombenergieoperator und $A_i$ ist Austauschoperator.
	\begin{align*}
		M \phi_i (\vec{r}_i) &= \sum_{j=1}^{N} \lambda_{ij} \phi_j (\vec{r}_i)
		& &(\text{Koeffizientenvergleich der Variation})
	\end{align*}
Diagonalisiere
	\begin{align*}
		\lambda &= \lambda^\dagger = (\lambda_{ij}) = (\lambda_{ij}^*) 
		= \U \in \U^\dagger 
		,& \epsilon &=
		\begin{pmatrix}
			\epsilon_1 & & 0 \\
			 & \ddots &  \\
			 0 &  & \epsilon_N
		\end{pmatrix} \\
		\phi_j &= \sum_i \U_{ji} \phi_i' \\
		& & \det
		\begin{pmatrix}
			\phi &  & \phi \\
			& \phi & \\
			\phi & & \ddots
		\end{pmatrix}
		&= \det
		\begin{pmatrix}
			\phi' & & \\
			& \ddots & \\
			& & \phi'
		\end{pmatrix} = \psi
	\end{align*}
	$\Rightarrow$
	\begin{empheq}[box = \boxed]{align*}
		(H^{(i)} + C_i - A_i) \phi_i' (\vec{r}_i) &= \epsilon_i \phi_i'(\vec{r}_i)
		& \braket{\phi_i'| \phi_j'} &= \delta_{ij}
	\end{empheq}
Hartre-Fock Gleichungen

	\begin{tabular}{l l}
		NB: & Ohne Slaterdeterminante ergibt sich $A_i = 0$ (Hartre-Gleichung) \\
		NB2: & Elektron $i$ bewegt sich im Feld des Kerns $(H^{(i)})$ und der restlichen $e^- (V_{ij})$, \\
		 & repräsentiert durch $C_i, A_i$ \\
		NB3: & $C_i - A_i$ muss für alle $i$ gleich sein, da die $e^-$ ununterscheidbar sind. \\
		 & $C_i$ und $A_i$ separat können aber von Zustand des Elektrons $i$ abhängen. \hfill \\
	\end{tabular} \\
		
Iteratives Lösungsverfahren

Wähle $\phi_i^{(0)}$, z.B. $H^{(i)} \phi_i^{(0)} = \epsilon_i \phi_i^{(0)}$
	\begin{align*}
		&\Rightarrow \left(H^{(i)} + C_i^{(0)} - A_i^{(0)}\right) \phi_i^{(1)}
		= \epsilon_i^{(1)} \phi_i^{(1)} \\
		&\Rightarrow \phi_i^{(1)} \curvearrowright C_i^{(1)} 
		,~ A_i^{(i)} \curvearrowright \phi_i^{(2)} \text{ etc.}
	\end{align*}
Grundzustandsenergie: \marginpar{23.11.2015}
	\begin{align*}
		E_0^{HF} &= \braket{\psi | H | \psi} 
		= \sum_{i = 1}^{N} \braket{\phi_i^{(n)} | H^{(i)}| \phi_i^{(n)}}
		+ \frac{1}{2} \sum_i \braket{\phi_i^{(n)} | C_i - A_i | \phi_i^{(n)}} \\
		&= \sum_i \epsilon_i - \frac{1}{2} \sum_i \braket{\phi_i^{(n)} | C_i - A_i | \phi_i^{(n)}}
	\end{align*}
Mit 
	\begin{align*}
		\sum_{i = 1}^{N} \braket{\phi_i | C_i | \phi_i}
		&= \sum_{\substack{i,j \\ i \neq j}}\int \diff^3 r_1 \diff^3 r_2 |\phi_i (\vec{r}_1)|^2 |\phi_j (\vec{r}_2)|^2 V_{ij} \\
		\sum_{i = 1}^{N} \braket{\phi_i | A_i | \phi_i}
		&= \sum_{\substack{i,j \\ i \neq j}} \int \diff^3 r_1 \diff^3 r_2 \phi_i^* (\vec{r}_1) \phi_j^* (\vec{r}_2) 
		V_{ij} \phi_i (\vec{r}_2) \phi_j (\vec{r}_1) \delta_{S_z^{(1)} S_z^{(2)}}
	\end{align*}