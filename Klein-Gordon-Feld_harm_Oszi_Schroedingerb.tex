\subsection{Klein-Gordon-Feld und harmonische Oszillatoren im Schrödingerbild (Operatoren sind zeitunabhängig)}
	\begin{align*}
		[\hat{q}_i, \hat{p}_j] &= i \delta_{ij},&
		[\hat{q}_i, \hat{q}_j] &= 0 = [\hat{p}_i, \hat{p}_j]
	\end{align*}
Analog für Felder:
	\begin{align*}
		[\hat{\phi}(\vec{x}), \hat{\pi}(\vec{y})] &= 
		i \delta^{(3)}(\vec{x}- \vec{y}),&
		[\hat{\phi}(\vec{x}), \hat{\phi}(\vec{y})] &= 0 = 
		[\hat{\pi}(\vec{x}), \hat{\pi}(\vec{y})]
	\end{align*}
$\hat{\phi}(\vec{x})$ ist Feldoperator.
	\begin{align*}
		\hat{H} = \hat{H} (\hat{\phi}, \hat{\pi})&: 
		\text{ Hamiltonoperator},&
		\hat{\Ha}&: \text{ Hamiltondichteopeator} \\
		\phi(\vec{x},t) &= \int \frac{\diff^3 p}{(2 \pi)^3} e^{i\vec{p}\vec{x}} \phi(\vec{p},t),& 
		\phi(\vec{x},t) \in \mathds{R}&\Rightarrow \phi^* (\vec{p}) = \phi (-\vec{p}) 
	\end{align*}
	\begin{align*}
		\left(
			\frac{\partial^2}{\partial t^2} - \vec{\nabla}^2 + m^2
		\right) \phi(\vec{x}, t) &= 0 \\
		\Rightarrow \left(
			\frac{\partial^2}{\partial t^2} + \vec{p}^2 + m^2
		\right) \phi(\vec{p}, t) &= 0
	\end{align*}
Harmonischer Oszillator mit $\omega_{\vec{p}} = \sqrt{\vec{p}^2 + m^2}$
	\begin{align*}
		H_{\text{HO}} &= \frac{\hat{\pi}^2}{2} + \frac{\omega_{\vec{p}}^2}{2} \hat{\phi}^2 : \text{ QM} \\
		[\hat{\phi}, \hat{\pi}] &= i \\
		\hat{\phi} &= \frac{1}{\sqrt{2 \omega_{\vec{p}}}} (\hat{a} + \hat{a}^\dagger),&
		\hat{\pi} &= - i \sqrt{\frac{\omega_{\vec{p}}}{2}}( \hat{a} - \hat{a}^\dagger) \\
		[a, a^\dagger] &= 1 & \Rightarrow
		H_{\text{HO}} &= \omega_{\vec{p}}(a^\dagger a + \frac{1}{2})
	\end{align*}
$\ket{n} \La (a^\dagger)^n \ket{0}$ ist Eigenzustand zu $H_{HO}$ mit Eigenwert $\omega (n + \frac{1}{2})$ 
	\begin{align*}
		\hat{\phi} (\vec{x}) &= 
		\int \frac{\diff^3 p}{(2 \pi)^3} \frac{1}{\sqrt{2 \omega_{\vec{p}}}}
		\left(	
			\hat{a}_{\vec{p}} \,e^{i\vec{p}\vec{a}} + \hat{a}^\dagger_{\vec{p}}\, e^{-i\vec{p}\vec{a}}
		\right) \\
		\phi (\vec{p},t) &= \frac{1}{2} (\phi(\vec{p},t) + \phi^*(-\vec{p},t)) \\
		\hat{\pi}(\vec{x}) &= \int \frac{\diff^3 p}{(2\pi)^3}
		\sqrt{\frac{\omega_{\vec{p}}}{2}} (\hat{a}_{\vec{p}} \,e^{i \vec{p} \vec{x}} -
		\hat{a}^\dagger_{\vec{p}} \,e^{- i \vec{p} \vec{x}}) 
	\end{align*}
Ab jetzt alles Operatoren, was wie Operator aussieht $\rightarrow$ lasse \,$\hat{}$\, weg.
	\begin{align*}
		\phi(\vec{x}) &= \int \frac{\diff^3 p}{(2\pi)^3} 
		\frac{1}{\sqrt{2\omega_{\vec{p}}}} 
		\left(
			a_{\vec{p}}\, e^{i \vec{p} \vec{x}} -
			a^\dagger_{\vec{p}} \,e^{- i \vec{p} \vec{x}}
		\right) \\
		\pi (\vec{x}) &= \int \frac{\diff^3 p}{(2 \pi)^3} (-i) \sqrt{\frac{\omega_{\vec{p}}}{2}} (a_{\vec{p}}e^{i \vec{p}\vec{x}} - a_{\vec{p}}^\dagger e^{-i \vec{p}\vec{x}})
		\\
		[a_{\vec{p}}, a_{\vec{p}\,'}^\dagger] &= (2\pi)^3 \delta^{(3)}(\vec{p}- \vec{p}\,') \\			
		\text{Nachrechnen: } \\ [\phi(\vec{x}), H(\vec{x})] &= i \delta^{(3)}(\vec{x}- \vec{x}\,')
	\end{align*}
$[a_{\vec{p}}, a_{\vec{p}\,'}^\dagger] = (2\pi)^3 \delta^{(3)} (\vec{p} - \vec{p}\,')$ \marginpar{25.01.2016}
	\begin{align*}
		\overbrace{\phi (\vec{x})}^{\mathclap{\text{Feld Operator}}} &= \int \frac{\diff^3 p}{(2 \pi)^3} \frac{1}{\sqrt{2 \omega_{\vec{p}}}} 
		\left(
			a_{\vec{p}} e^{i \vec{p} \vec{x}} + a_{\vec{p}}^\dagger e^{-i\vec{p} \vec{x}}
		\right) 
		= \int \frac{\diff^3 p}{(2\pi)^3} \frac{1}{\sqrt{2 \omega_{\vec{p}}}} (a_{\vec{p}} + a_{-\vec{p}}^\dagger) e^{i \vec{p} \vec{x}}
	\end{align*}
Kanonischer Impulsoperator (hier fehlt noch ein Wort):
	\begin{align*}
		\pi (\vec{x}) &= \int \frac{\diff^3 p}{(2\pi)^3} (-i) \sqrt{\frac{\omega_{\vec{p}}}{2}} 
		\left(
			a_{\vec{p}} e^{i\vec{p} \vec{x}} - a^\dagger_{\vec{p}} e^{-i\vec{p} \vec{x}}
		\right) 
		= \int \frac{\diff^3 p}{(2\pi)^3} (-i) \sqrt{\frac{\omega_{\vec{p}}}{2}} 
		(a_{\vec{p}} - a_{-\vec{p}}^\dagger) e^{ipx} \\
		[\phi(\vec{x}), \pi(\vec{x}\,')] &=
		\int \frac{\diff p^3 \diff p\,'^3}{(2\pi)^6} (-i) \sqrt{\frac{\omega_{p\,'}}{4 \omega_p}} 
		\underbrace{
		\left[
			(a_{\vec{p}} + a_{-\vec{p}}^\dagger), (a_{\vec{p}\,'} - a_{-\vec{p}\,'}^\dagger)
		\right]}_{\mathclap{
				[a_{\vec{p}}^\dagger, a_{\vec{p}\,'}] - [a_{\vec{p}}, a_{-\vec{p}\,'}^\dagger] = -2 (2\pi)^3 \delta (\vec{p}+\vec{p}\,')
			}}
		e^{i\vec{p}\vec{x}} e^{i p\,' x\,'} \\
		&= \int \frac{\diff^3 p}{(2\pi)^3} (+i) e^{i \vec{p} (\vec{x}- \vec{x}\,')} = 
		i\delta^3(\vec{x}- \vec{x}\,')
	\end{align*}
	\begin{align*}
		H &= \int \diff^3 x 
		\left(
			\frac{\pi^2}{2} + \frac{1}{2} (\vec{\nabla} \phi)^2 + \frac{m^2}{2} \phi^2
		\right) \\
		&= \int \diff^3 x \int \frac{\diff^3 p \diff^3 p\,'}{(2\pi)^6} e^{i (\vec{p} + \vec{p}\,')\vec{x}} 
		\left[
			- \frac{\sqrt{\omega_{\vec{p}} \omega_{\vec{p}\,'}}}{2} 
			(a_{\vec{p}} - a_{-\vec{p}}^\dagger) \cdot (a_{\vec{p}\,'} - a_{-\vec{p}\,'}^\dagger)
		\right.\\
		&\left.
			+ \frac{m^2 - \vec{p} \cdot \vec{p}\,'}{2\sqrt{\omega_p \omega_{p\,'}}}
			(a_{\vec{p}} + a_{-\vec{p}}^\dagger) \cdot (a_{\vec{p}\,'} + a_{-\vec{p}\,'}^\dagger)
		\right] \\
		&= \int \frac{\diff^3 p}{(2 \pi)^3} \omega_{\vec{p}} 
		\left(
			(a^\dagger_{\vec{p}} a_{\vec{p}}) + \frac{1}{2} [a_{\vec{p}}, a^\dagger_{\vec{p}}]
		\right)
	\end{align*}
$\frac{1}{2}$ für jeden harmonischen Oszilator (jedes der unendlichen $\vec{p}$)
\\ 
Wir können nur Energiedifferenzen bestimmen (Ausnahme Gravitation):
 
Definiere Nullpunktsenergie $E_0 = 0$
	\begin{align*}
		\Rightarrow \boxed{H = \int \frac{\diff^3 p}{(2 \pi)^3} \omega_{\vec{p}}\, a_{\vec{p}}^\dagger \,a_{\vec{p}}}
	\end{align*}
	\begin{align*}
		[H, a_{\vec{p}}^\dagger] &= \omega_{\vec{p}}\, a_{\vec{p}}^\dagger ,&
		[H, a_{\vec{p}}] &= - \omega_{\vec{p}} \,a_{\vec{p}}
	\end{align*}
Grundzustand (Vakuum): 
$\ket{0}, ~a_{\vec{p}}\ket{0} = 0 ~\text{für alle }p^0 \text{ ist }E_0 = 0$
	\begin{align*}
		a_{\vec{p}}^\dagger \,a_{\vec{p}}^\dagger \ldots \ket{0} \text{ ist Eigenzustand von } H \text{ mit Energie: }  E= E\vec{p} + E\vec{q} + \ldots
	\end{align*}
Physikalischer Gesamtimpuls: $\vec{P}$
	\begin{align*}
		\vec{P} &= - \int \diff^3 x \overbrace{\pi (\vec{x})}^{\mathclap{\text{Impulsdichte Operator}}} \vec{\nabla} \underbrace{\phi(\vec{x})}_{\mathclap{\text{Feldoperator}}}
		= \int \frac{\diff^3 p \diff^3 p\,'}{(2\pi)^6} \frac{1}{2} (a_{\vec{p}} - a_{-\vec{p}}^\dagger) (- \vec{p}\,') (a_{\vec{p}} + a_{-\vec{p}\,'}^\dagger) e^{-i\vec{p} \vec{x}} e^{i \vec{p}\,' \vec{x}} \ldots \\
		&= \int \frac{\diff^3 p}{(2\pi)^3} \vec{p} 
		\left(
			a_{\vec{p}}^\dagger \,a_{\vec{p}} + \left(\frac{1}{2} [a_{\vec{p}}, a_{\vec{p}}^\dagger] \right)
		\right)
	\end{align*}
$a_{\vec{p}}^\dagger a_{\vec{q}}^\dagger \ket{0}$ hat Gesamtimpuls $\vec{p} + \vec{q}$

Anregung $a_{\vec{p}}^\dagger \ket{0}$ wird ``Teilchen\,'\,' genannt da, es sich um diskrete Objekte handelt mit: \marginpar{bei diskret bin ich mir nicht sicher}
	\begin{align*}
		E_{\vec{p}} &= \omega_{\vec{p}} = \sqrt{\vec{p}^2 + m^2} 
	\end{align*}
	\begin{align*}
		a_{\vec{p}}^\dagger \ket{0} &: \text{1-Teilchenzustand} \\
		a_{\vec{p}}^\dagger\, a_{\vec{q}}^\dagger \ket{0} &: \text{2-Teilchenzustand etc}
	\end{align*}
	\begin{align*}
		N_{\vec{p}} &= a_{\vec{p}}^\dagger \,a_{\vec{p}} & 
		N_{\vec{p}} \ket{\psi} &= \underbrace{\omega_{\vec{p}}}_{\mathclap{\text{Anzahl der Teilchen mit Impuls }\vec{p}}} \ket{\psi} \\
		a_{\vec{p}}^\dagger\, a_{\vec{q}}^\dagger \ket{0} &= a_{\vec{q}}^\dagger\, a_{\vec{p}}^\dagger \ket{0} &
		\text{weil } [a_{\vec{p}}^\dagger, a_{\vec{q}}^\dagger] &= [a_{\vec{p}}, a_{\vec{q}}] = 0
	\end{align*}
Zustand ist symmetrisch im Bezug auf Vertauschen von Teilchen. 
\\
$\Rightarrow \phi$ Feldoperator für Bosonen (Bose-Einstein-Statistik) 

NB: Für Fermionen gilt: $ \{a_{\vec{p}}^\dagger, a_{\vec{p}}^\dagger\} = 0 = \{a_{\vec{p}}, a_{\vec{p}}\}$ 
	\begin{align*}
		\{a_{\vec{p}}, a_{\vec{p}\,'}^\dagger\} &= (2 \pi)^3 \delta (\vec{p} - \vec{p}\,') 
		& \text{mit Antikommutator} \{A, B\} &= A\, B + B\, A
	\end{align*}
$\Rightarrow n_p > 1$ nicht möglich.
\\
\underline{Normierung:} $\braket{0 | 0} = 1$ 

Einteilchenzustand mit Impuls: %$\vec{p} : \ket{\vec{p}} \propto a_{\vec{p}}^\dagger \ket{0}$
	\begin{align*}
		\vec{p} : \ket{\vec{p}} \propto a_{\vec{p}}^\dagger \ket{0}
	\end{align*}

Naive Normierung: %$\braket{\vec{p} | \vec{q}} = (2 \pi)^3 \delta^{(3)}(\vec{q} - \vec{p})$
	\begin{align*}
		\braket{\vec{p} | \vec{q}} = (2 \pi)^3 \delta^{(3)}(\vec{q} - \vec{p})
	\end{align*}

ist nicht relativistisch invariant.
\\
Beispiel Lorentzboost mit Geschwindigkeit $v$ in 3-Richtungen:
	\begin{align*}
		p_3\,' &= \gamma (p_3 + v E) & p_0\,'&= E = \gamma (E+ vp_3),& &KANN ICH NICHT LESEN
	\end{align*}
Jetzt $x_0$ einfache Nullstelle
	\begin{align*}
		\delta (f(x) - f(x_0)) &= \frac{1}{|f\,'(x)|} \delta (x - x_0) \\
		\delta^{(3)} (\vec{p}- \vec{q}) &= \delta^{(3)} (\vec{p}\,'- \vec{q}\,') 
		= \delta^{(3)}(\vec{p}\,' - \vec{q}\,') \gamma 
		\left(
			1 + v \frac{\diff E}{\diff p_3} 
		\right) \\
		&= \delta^{(3)} (\vec{p}\,' - \vec{q}\,') \frac{\gamma}{E} (E + v p_3) 
		= \delta^{(3)} (\vec{p}\,' - \vec{q}\,') \frac{E\,'}{E}
	\end{align*}
$\Rightarrow 2E \delta^{(3)} (\vec{p} - \vec{q})$ ist Lorentzinvariant.
\\
\underline{NB:} 
	\begin{align*}
		\int \frac{\diff^3 p}{(2\pi)^3} \frac{1}{2 E_{\vec{p}}} &=
		\int \frac{\diff^4 p}{(2 \pi)^4} \left.\delta (p^2 - m^2)\right|_{p^0 = E_{\vec{p}}}
	\end{align*} 
\marginpar{ich konnte nicht genau erkennen, auf was eingeschränkt}
ist auch explizit Lorentzinvariant.
	\begin{align*}
		\Rightarrow \ket{\vec{p}} &\coloneqq \sqrt{2 E_{\vec{p}}}\, a_{\vec{p}}^\dagger \ket{0} \\
		&\Rightarrow \braket{\vec{p} | \vec{q}} = 2 E_{\vec{p}} \,(2 \pi)^3 \delta^{(3)} (\vec{q} - \vec{p}) \\
		&\phi (x) = \int \frac{\diff^3 p}{(2\pi)^3} \frac{\sqrt{2 E_p}}{2 E_p} 
		\left(
			a_{\vec{p}} \, e^{i\vec{p} \vec{x}} + a_{\vec{p}}^\dagger \, e^{-i\vec{p} \vec{x}}
		\right) \checkmark 
	\end{align*}
Lorentztrafo von $\ket{\vec{p}}$: 
	\begin{align*}
		p\,'^\mu &= \Lambda^\mu_m p^\mu & \text{defniniere } \vec{p}\,' &= \Lambda^\mu_m (\vec{p}) ~(\text{massenunabhängig}) 
	\end{align*}
Auf dem Hilbertraum ist die Lorentz Trafo eine unitäre Transformation $\U (\Lambda)$ 
	\begin{align*}
		\U(\Lambda) \ket{\vec{p}}&= \U(\Lambda) \sqrt{2 E_{\vec{p}}} \,a_{\vec{p}}^\dagger \, \U^{-1}(\Lambda) \underbrace{(\U(\Lambda) \ket{0})}_{\mathclap{= \ket{0}}} \\
		&= \ket{\Lambda_m(\vec{p})} = a_{\Lambda_m (\vec{p})}^\dagger \sqrt{2 E_{\vec{p}\,'}} \ket{0} \\
		\U(\Lambda) \, a_{\vec{p}}^\dagger \, \U^{-1} (\Lambda) &= 
		\sqrt{\frac{E_{\Lambda_m (\vec{p})}}{E_{\vec{p}}}} a^\dagger_{\Lambda_m (\vec{p})} 
	\end{align*}
	\begin{align*}
		\mathds{1}_{\text{1-Teilchen}} &= \int \frac{\diff^3 p}{(2\pi)^3} \ket{\vec{p}} \frac{1}{2 E_{\vec{p}}} \bra{\vec{p}} \\
		\phi(x) \ket{0} &= \int \frac{\diff^3 p}{(2\pi)^3} \frac{1}{2 E_{\vec{p}}} e^{-i\vec{p} \vec{x}} 
		\underbrace{\sqrt{2 E_{\vec{p}}} a^\dagger \ket{0}}_{\mathclap{\ket{\vec{p}}}} \\
		&= \int \frac{\diff^3 p}{(2 \pi)^3} \frac{1}{2 E_{\vec{p}}} e^{-i\vec{p} \vec{x}} \ket{\vec{p}} \text{ Fouriertrafo von } \ket{\vec{p}}
	\end{align*}
$\Rightarrow \phi (\vec{x})$ erzeugt ein Teilchen auf der Position $\vec{x}$. 
	\begin{align*}
		\braket{0 | \phi(\vec{x}) | 0} &= 
		\braket{0 | \int \frac{\diff^3 p}{(2 \pi)^3} \frac{1}{\sqrt{2 E_{\vec{p}}}} 
			\left(
				a_{\vec{p}\,'} e^{+i \vec{p}\,' \vec{x}} + a_{\vec{p}\,'}^\dagger e^{-i\vec{p}\,' \vec{x}}	
			\right) \sqrt{2 E_{\vec{p}}} \,a_{\vec{p}}^\dagger\,
		| 0} \\
		&= e^{+ i \vec{p} \vec{x}}
	\end{align*}
Analog: $ \braket{\vec{x} | \vec{p}} = e^{+ i \vec{p} \vec{x}}$ in normaler QM