\subsection{Kopplung der Dirac-Gleichung an elektromagnetische Felder}
``Rezept'' aus QM1 (2)
	\begin{align*}
		\vec{p} &\mapsto \vec{p} - \frac{q}{c} \vec{A},&
		q &= -e \text{ für Elektronen}
	\end{align*}		
	\begin{align*}
		&\boxed{
				H = c\vec{\alpha}
				\left(\vec{p} - \frac{q}{c} \vec{A}\right)
				+ \beta m c^2 + q \Phi
			} \\
		&\Rightarrow
		\boxed{
				\left(
					i \hbar \gamma^\mu \partial_\mu - \frac{q}{c} \gamma^\mu A_\mu - mc
				\right) \psi = 0
			} 
		& \gamma_\mu &= \beta \vec{\alpha},& \beta &= \gamma_0
	\end{align*}
	\begin{align*}
		A^\mu &= 
		\begin{pmatrix}
			\Phi \\ \vec{A}
		\end{pmatrix}
		& \vec{\alpha} &=
		\begin{pmatrix}
			0 & \vec{\sigma} \\
			\vec{\sigma} & 0
		\end{pmatrix}
		& \beta &= 
		\begin{pmatrix}
			\mathds{1} & 0 \\
			0 & \mathds{1}
		\end{pmatrix}
		& \psi &=
		\begin{pmatrix}
			\phi \\ \chi
		\end{pmatrix}
	\end{align*}
	\begin{align*}
		\left(
			i\hbar \frac{\partial}{\partial t} - q \Phi - mc^2
		\right) \phi 
		&= (c\vec{p} - q \vec{A}) \vec{\sigma} \chi \\
		\left(
			i \hbar \frac{\partial}{\partial t} - q \Phi + mc^2
		\right) \chi
		&= (c \vec{p} - q \vec{A}) \vec{\sigma} \phi
	\end{align*}
Stationäre Gleichungen $\left(i\hbar \frac{\partial \chi}{\partial t} = E \chi,~ i\hbar \frac{\partial \phi}{\partial t} = E \phi \right)$.
	\begin{align*}
		\left(E - q\Phi - mc^2\right) \phi 
		&= (c \vec{p} - q \vec{A}) \cdot \vec{\sigma} \chi \\
		\left(E - q \Phi + mc^2\right) \chi
		&= (c \vec{p} - q \vec{A}) \cdot \vec{\sigma} \phi
	\end{align*}
Was heißt
	\begin{align*}
		\vec{A} \vec{\sigma} \phi =
		\underbrace{
				\left(A_x\sigma_x + A_y\sigma_y + A_z\sigma_z\right)
			}_{\substack{
					\begin{pmatrix}
						A_z & A_x - iA_y \\
						A_x + iA_y & -A_z
					\end{pmatrix}
				}}
		\underbrace{\phi}_{\mathclap{
					\begin{pmatrix}
						\phi_1 \\ \phi_2
					\end{pmatrix}
				}}
	\end{align*}
Nichtrelativistischer Grenzfall:
	\begin{align*}
		E &= mc^2 + E' ,& 
		|E'| &\approx \frac{\vec{p}}{2m} + \ldots \ll mc^2 ,&
		|q\Phi| &\ll mc^2 \\
		(E' - q \Phi) \phi &\approx (c \vec{p} - q \vec{A}) \vec{\sigma} \chi \\
		2mc^2 \chi &\approx (c \vec{p} - q \vec{A}) \vec{\sigma} \phi \\
		(E' - q\Phi) \phi &= 
		\frac{1}{2mc^2} \left[(c \vec{p} - q \vec{A}) \vec{\sigma}\right]^2 \phi  
	\end{align*}
Nebenrechnung:
	\begin{align*}
		(\vec{\alpha} \cdot \vec{\sigma}) (\vec{b} \cdot \vec{\sigma}) 
		&= (\vec{a} \cdot \vec{b}) \mathds{1} + i (\vec{a} \times \vec{b}) \cdot \vec{\sigma}
	\end{align*}
	\begin{align*}
		(E'- q\Phi) \phi &=
		\left[\frac{1}{2m} (\vec{p} - \frac{q}{c} \vec{A})^2 - \frac{q \hbar}{2mc} \vec{\sigma} \cdot \vec{B}\right] \phi
	\end{align*}
Hier fehlt eine skurile Rechnung, bei der jeder den Überblick so verloren hatte, dass ich sie nicht mitgeschrieben hab.
	\begin{align*}
		&\boxed{\left[
			\frac{1}{2m} (\vec{p} - \frac{q}{c} \vec{A})^2 + q\Phi -\frac{q}{mc} \vec{S} \cdot \vec{B}
		\right] \phi
		= E\phi}
		&\text{mit }
		\vec{S} &= \frac{\hbar}{2} \vec{\sigma} ~(\text{Spinoperator}) \\
		&\text{Pauligleichung mit }
		-\frac{q}{mc} \vec{S} \cdot \vec{B} = g \frac{e}{2mc} \vec{S} \cdot \vec{B}
	\end{align*}
Gyromagnetischer Faktor $g=2$
	\begin{align*}
		\vec{L} + g \vec{S} = \vec{L} + 2\vec{S} = \vec{J} + \vec{S}
	\end{align*}
Es existieren Korrekturen zu $g=2$ (Quantenelektrodynamik)\\
$4\times4$ Spin matrizen:
	\begin{align*}
		\vec{\Sigma} &=
		\begin{pmatrix}
			\vec{\sigma} & 0 \\
			0 & \vec{\sigma} 
		\end{pmatrix}
	\end{align*}
Spinoperator, der auf Dirac-Spinoren wirkt: $\vec{S} = \frac{\hbar}{2} \vec{\Sigma}$
	\begin{align*}
		\text{Diracgleichung } &\Rightarrow
		\begin{aligned}
			&1.\text{ Spin} \\
			&2.\text{ Positron} \\
			&3.\text{ Pauli} \\
			&4. ~g=2 \\
		\end{aligned}
	\end{align*}
	\begin{align*}
		[S_j,S_k] &= i\epsilon_{jk \ell} S_\ell \\
		\vec{S}^2 &= 3 \frac{\hbar^2}{4} =
		S(S + 1) \hbar^2 \text{ mit } S=\frac{1}{2}
	\end{align*}
Bahndrehimpuls $\vec{L} = \vec{r} \times \vec{p}$
	\begin{align*}
		H &= c \vec{\alpha} \vec{p} + \beta mc^2 & &(A^\mu = 0) \\
		[\vec{L}, H] &= i\hbar c \vec{\alpha} \times \vec{p}
		& [\vec{S}, H] &= -i\hbar c \vec{\alpha} \times \vec{p} \\
		\Rightarrow [\vec{J}, H] &= 0
		& \text{mit Gesamtdrehimpuls } 
		\vec{J} &= \vec{L} + \vec{S} = \vec{L} \otimes \mathds{1}_4 + \mathds{1}_{2\ell} \otimes \vec{S}
	\end{align*}
	\begin{align*}
		S_z u^{(1)} &= \frac{\hbar}{2} u^{(1)} &
		S_z u^{(3)} &= \frac{\hbar}{2} u^{(3)} &
		S_z u^{(2)} &= -\frac{\hbar}{2} u^{(2)} &
		S_z u^{(4)} &= -\frac{\hbar}{2} u^{(4)}
	\end{align*}
	\begin{align*}
		\phi &= 
		\begin{pmatrix}
			u^{(1)} \\
			u^{(3)}
		\end{pmatrix}
	\end{align*}
(Teilchen für $|E| - mc^2$ klein) $u^{(1)}$: spin ``up'', $u^{(3)}$: spin ``down''
