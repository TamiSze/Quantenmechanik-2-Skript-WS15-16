\subsection{Das Pauliprinzip}
Beispiel: Atom mit $N$ Elektronen $e^-$.
	\begin{align*}
		H &= \sum_{i=1}^N H^{(i)} ,& 
		H^{(i)} &= - \frac{\vec{p}\,^{(i)2}}{2m} + V(\vec{r}\,^{(i)}) 
	\end{align*}
Hierbei wurde die $e^- -e^- -$ Wechselwirkung ungerechtfertigerweise vernachlässigt.
	
stationäre Schrödinger Gleichung:
	\begin{align*}
		H \phi_{\vec{\alpha}} &= E_{\vec{\alpha}} \phi_{\vec{\alpha}} 
		&\text{mit } \phi_{\vec{\alpha}} &= \prod_{i=1}^N \phi_{\alpha_i} \\
		H^{(i)} \phi_{\alpha_i} &= E_{\alpha_i} \phi_{\alpha_i} 
		&E_{\vec{\alpha}} &= \sum_{i=1}^N E_{\alpha_i} \\
		\alpha_i &= (n_i, \ell_i , m_i) 
		&\text{oder mit } &(\vec{p}+e \vec{A})^2 
		\text{ Pauligleichung} \\
		\alpha_i &= (n_i, j_i, m_i , \ell_i ,(s_i))
	\end{align*}
und $s_i = \frac{1}{2}$
	\begin{align*}
		n_i &: \text{Hauptquantenzahlen} \\
		j_i &: \text{Drehimpulsquantenzahl } ( \text{Gesamtdrehimpuls } \vec{J}_i = \vec{L}_i + \vec{S}_i) \\
		m_i &: \text{Magnetsiche Quantenzahl } (\vec{J}_i)_z \ket{\phi_{\alpha_i}} = m_i \ket{\phi_{\alpha_i}} \\
		\ell_i &: \text{Bahndrehimpulsquantenzahl}
	\end{align*}
Grundzustand:
	\begin{align*}
		\alpha_i = (1, \frac{1}{2}, \pm \frac{1}{2}, 0)
	\end{align*}
Pauliverbot (Ausschließungsprinzip):
		
Jeder Einteilchenzustand $\phi_\alpha$ kann höchstens mit einem $e^-$ besetzt sein. \\
Pauliprinzip (Dirac + Heisenberg):
		
Die Wellenfunktion eines Systems von Elektronen ist total antisymmetrisch. \\
Definiere Permutation
	\begin{align*}
		\pi &\in S_N ,~ \sigma (\pi) = \mathrm{sign}(\pi) \in \{+1, -1\} \\	
		\sigma &: \text{Signum der Permutation}\\
		\sigma (\pi) &=
		\left\{
			\begin{aligned}
				&+1 \text{ gerade} ,& &\text{gerade Anzahl von paarweise VerIRGENDWAS} \\
				& & &\text{führt } (1,2,3, \ldots) \text{ in } \prod (1,2,3,\ldots) \text{ über}\\
				&-1 \text{ ungerade} ,& &\text{ungerade Anzahl} (\ldots)
			\end{aligned}
		\right.
	\end{align*}
\underline{Antisymmetrisierungsoperator}:
	\begin{align*}
		A &= \frac{1}{\sqrt{N!}} \sum_{\pi \in S_N} \sigma (\pi) \cdot \pi 
		&\Rightarrow (A \phi) (\vec{r}_1, \ldots, \vec{r}_N) 
		&= \frac{1}{\sqrt{N!}} \sum_{\pi \in S_N} \sigma(\pi)
		\phi \left((\vec{r}_{\pi(1)}, \vec{r}_{\pi(2)}, \ldots ,\vec{r}_{\pi(N)}\right)   
	\end{align*}
	\begin{align*}
		A \phi_{\vec{\alpha}} (\vec{r}_1, \ldots, \vec{r}_N)
		&= \frac{1}{\sqrt{N!}} 
		\det
		\begin{pmatrix}
			\phi_{\alpha_1}(\vec{r}_1, \sigma_1) & \cdots & \phi_{\alpha_1}(\vec{r}_N, \sigma_N) \\
			\vdots & \ddots & \vdots\\
			\phi_{\alpha_N}(\vec{r}_1, \sigma_1) & \cdots & \phi_{\alpha_N}(\vec{r}_N, \sigma_N)
		\end{pmatrix}
	\end{align*}
Wdh: 1-Teilchen Wellenfunktionen (Fermionen)\marginpar{16.11.2015}
	\begin{align*}
		\phi_{\alpha_1}(\vec{r}_1, \sigma_1), \ldots, \phi_{\alpha_N}(\vec{r}_N, \sigma_N)
	\end{align*}
Das $\sigma$ zeigt Spin an.
	\begin{align*}
		\phi_{\alpha_1} &= 
		\begin{pmatrix}
			\phi_{+, \alpha_1 (\vec{r}_1)} \\
			\phi_{-, \alpha_1 (\vec{r}_1)}
		\end{pmatrix}
	\end{align*}
hier kommt irgendwas von flavour, was uns nur verwirren kann.
	
Fermionische Gesamtwellenfunktion ist total antisymmetrisch
	\begin{align*}
		\text{Spin } &\in \left\{ 0, \frac{1}{2}, 1 , \frac{3}{2}, 2 , \ldots \right\} 
		& (\text{Pauliprinzip})
	\end{align*}
wobei $0, 1 , 2$ Bosonen sind und $\frac{1}{2}, \frac{3}{2}$ Fermionen sind.

Antisymmmetris... siehe oben
	
Beispiel: $N = 2$
	\begin{align*}
		\psi_{\alpha_1 \alpha_2} (\vec{r}_1, \sigma_1; \vec{r}_2, \sigma_2)
		&= A \overbrace{(\phi_{\alpha_1 \alpha_2} (\vec{r}_1, \sigma_1; \vec{r}_2, \sigma_2))}^{\mathclap{\phi_{\alpha_1} (\vec{r}_1, \sigma_1) \cdot \phi_{\alpha_2}(\vec{r}_2, \sigma_2)}} \\
		&= \frac{1}{\sqrt{2}}
		\left(\phi_{\alpha_1} (\vec{r}_1, \sigma_1) \cdot \phi_{\alpha_2}(\vec{r}_2, \sigma_2)
		- \phi_{\alpha_1} (\vec{r}_2, \sigma_2) \cdot \phi_{\alpha_2}(\vec{r}_1, \sigma_1)\right)
	\end{align*}
Antisymmetrische Gesamtwellenfunktion von $2 e^-$ im Zustand $\alpha_1$ und $\alpha_2$ 
	\begin{align*}
		\alpha_1 &= \alpha_2 \Rightarrow \psi = 0 ~(\text{Pauliverbot})
	\end{align*}
	\begin{align*}
		\text{Bosonen } \longleftrightarrow& 
		\text{ ganzzahliger Spin (Beispiel: Photon)} \\
		&\Rightarrow \text{ Total symmetrische Wellenfunktion}\\
		\text{Fermionen } \longleftrightarrow& 
		\text{ halbzahliger Spin (Beispiel: Elektron)} \\
		&\Rightarrow \text{ Total antisymmetrische Wellenfunktion}
	\end{align*}