\subsection{Reine Zustände} 
Wiederholung: Quantenmechanik Regeln
	\begin{enumerate}[1.]
		\item Vorhersagen von Messergebnissen an einem ansonsten isolierten System sind probabilistisch. In Situationen, in denen die \underline{maximal mögliche} Information bekannt ist, kann diese dargestellt werden durch einen Vektor/Strahl $\ket{\psi}$ in einem komplexen Hilbertraum $\Hil$.
		\item Jede Observable $A$ kann dargestellt werden als selbstadjungierter Operator $\hat{A}$, welcher auf Vektoren in $\Hil$ wirkt.
		
		(NB: $\vec{x}$ ist z.B. keine echte Observable, da nicht-normierbarer Eigenvektor (uneigentlich). Es existiert auch kein Messinstrument, mit welchem sich $\vec{x}$ beliebig genau bestimmen lässt.)
		\item Erwartungswert einer Messung von $A$: $\erw{A}_\psi = \braket{\psi | A | \psi}$
		\label{Regel3}
		\item In einem \underline{abgeschlossenen} System ist
			\begin{align*}
				i \hbar \frac{\diff}{\diff t} \ket{\psi (t)} = \hat{H} \ket{\psi(t)}
			\end{align*}
		$\hat{H}$ : Hamiltonoperator
		\begin{enumerate}[(a)]
			\item Einzigen möglichen Messergebnisse von $A$ sind Eigenwerte (EW) $a_i$ von $\hat{A}$.
			\item Wahrscheinlichkeit
				\begin{align*}
					&W(A = a_n \ket{\psi}) = \braket{\psi | \hat{P}_n | \psi} \\
					\text{mit}~
					&\boxed{\hat{P}_n = \sum_{j = 1}^{dim (n)} \ket{a_n, j} \bra{a_n, j}}
					~\text{: Projektor}
				\end{align*}
			$dim$ von Eigenwert zum EW $a_n$ (Entartung)
			
			(NB: $\braket{\psi | \hat{P}_n | \psi} = \sum_{j= 1}^{dim(n)}|\braket{a_n, j | \psi}|^2$)
			
			Beweis: Statistsisches Mittel
				\begin{align*}
					\erw{F(A)} = \sum_m F(\alpha_m) W(A = \alpha_m)
				\end{align*}
				
			Beispiel für $\mathscr{F}$:
				\begin{align*}
					\chi_r (A) &= 
					\left\{
					\begin{aligned}
						1&, A = r \\
						0&, A \neq r
					\end{aligned}
					\right. \\
					\Rightarrow W (A = r) &= \erw{\chi_r (A)} \\
					\text{QM: } \\
					\erw{\mathscr{F}(A)}_\psi &= \sum_m \mathscr{F}(\alpha_m) W(A = \alpha_m; \ket{\psi}) \\
					\erw{\chi_r(A)} &\underset{\text{Regel \ref{Regel3}}}{=}
					\braket{\psi | \chi_r(\hat{A}) | \psi} = \sum_m \chi_r (\alpha_m) W(A = \alpha_m ; \ket{\psi}) \\
					\chi_r (\hat{A}) &= \sum_m \chi_r (a_m) \hat{P}_m \\
					\Rightarrow \sum_m \chi_r (a_m) \braket{\psi | \hat{P}_m | \psi} 
					&= \sum_m \chi_r (\alpha_m) W(A = \alpha_m; \psi) \\
					\braket{\psi | \hat{P}_m | \psi} &=
					\left\{
					\begin{aligned}
						\braket{\psi | \hat{P}_m | \psi} &, a_m = r \\
						0 &, \text{ sonst}
					\end{aligned}
					\right. \\
					W(A = \alpha_m; \psi) &=
					\left\{
					\begin{aligned}
						W(A = \alpha_m; \psi) &, \alpha_m = r \\
						0 &, \text{ sonst}
					\end{aligned}
					\right.
				\end{align*}
				$\Rightarrow$
				\begin{enumerate}[(a)] 
					\item $\alpha_m = a_m$
					\item $\braket{\psi | \hat{P}_m | \psi} = 
					W(A = a_m ; \ket{\psi}) = 
					\sum_{j = 1}^{dim(m)} |\braket{a_m, j | \psi}|^2$
				\end{enumerate}
				\begin{align*}
				\mathds{1} \underset{\text{weil } \ket{a_m, j} \text{ Basis von }\Hil}{=}
				\sum_m \hat{P}_m \Rightarrow
				1 = \braket{\psi | \psi} &= \sum_m \braket{\psi | \hat{P}_m | \psi} 
				= \sum_m W(A = a_m ; \ket{\psi}) \\
				\sum \text{ Wahrscheinlichkeiten} &= 1 \checkmark \\
				\hat{P}_m \text{ist ``Projektionsoperator''}& \\
				\hat{P}_m^2 = \sum_{j, k = 1}^{dim(m)} \ket{a_m, j} \braket{a_m j | a_m, k} \bra{a_m, k} &= \hat{P}_m
				\end{align*}
				$\hat{P}^2 = \hat{P} \Rightarrow$ Nur Eigenwerte 0 oder 1 sind möglich
		\end{enumerate}
	\end{enumerate}	
