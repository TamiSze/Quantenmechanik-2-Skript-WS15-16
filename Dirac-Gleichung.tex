\section{Relativistische Quantenmechanik}
\subsection{Die Dirac-Gleichung}
NR QM:
	\begin{align*}
		H &= \frac{p^2}{2m} 
		& \vec{p} &\mapsto \hat{\vec{p}} = -i \hbar \vec{\nabla} \\
		& & E &\mapsto H = i \hbar \frac{\partial}{\partial t} \\
		i \hbar \frac{\partial}{\partial t} \ket{\psi} 
		&= \frac{\hat{p}^2}{2m} \ket{\psi} 
	\end{align*}
Spezielle Relativitätstheorie:
	\begin{align*}
		H^2 &= p^2 c^2 + m^2 c^4 
		&\Rightarrow H &= \sqrt{p^2c^2 +m^2 c^4} 
		= mc^2\left(1 + \frac{\vec{p}^2}{2m} - \frac{(\vec{p}^2)^2}{8 m^4 c^4} + \ldots\right)
	\end{align*}
Relativistische Schrödinger Gleichung (?):
	\begin{align*}
		i \hbar \frac{\partial}{\partial t} \ket{\psi} 
		= \sqrt{p^2c^2 +m^2 c^4} \ket{\psi}
	\end{align*}
Ortsraum:
	\begin{align*}
		i \hbar \frac{\partial}{\partial t} \ket{\psi} 
		= mc^2 \left(1 - \frac{\vec{\hbar^2}}{2mc^2}\vec{\nabla}^2 - \frac{\hbar^2(\vec{\nabla}^2)^2}{8 m^4 c^4}\right)
	\end{align*}
	\begin{enumerate}
		\item unsymmetrisch im Raum und Zeit (nicht forminvariant unter Lorentz-Transformation)
		\item Hohe Ableitungen
	\end{enumerate}
Zwei Möglichkeiten der relativen Verallgemeinerung der (freien) Schrödingergleichung 
\marginpar{?? punkt 2 kam nie vor}
	\begin{align*}
		H^2 \ket{\psi} &= (c^2 \hat{p}^2 + m^2 c^4) \ket{\psi} \\
		-\hbar^2 \frac{\partial^2 \psi}{\partial t^2}
		&= (-\hbar^2 c^2 \vec{\nabla}^2 + m^2 c^4) \psi (\vec{r} , t) \\
		&\Rightarrow
		\boxed{
			\left(
			\frac{1}{c^2} \frac{\partial^2 \psi}{\partial t^2} - \vec{\nabla}^2 
			+ \left(\frac{mc}{\hbar}\right)^2
			\right)
			\psi (\vec{r} , t) = 0
		}
	\end{align*} 
Klein-Gordon-Gleichung, und $\frac{1}{c^2} \frac{\partial^2 \psi}{\partial t^2} - \vec{\nabla}^2 = \Box$ d'Alembertoperator.

Kein ``Spin'' Möglich für skalare Teilchen

Spin $=0$: Higgs, Pionen, $\alpha-$Teilchen etc.
\\
Dirac: erste Ableitun in $t \Rightarrow$ sollte linear in $p_x, p_y, p_z$ sein, damit forminvariant unter Lorentz-Transformation.

Ansatz:
	\begin{align*}
		H^2 &= p^2 c^2 + m^2 c^4 = c^2 (\vec{\alpha} \cdot \vec{p} + \beta m c)^2\\
		&= c^2 
		\left[
			\alpha_x^2 p_x^2 + \alpha_y^2 p_y^2 +\alpha_z^2 p_z^2
			+ \beta^2m^2c^2 + \alpha_x \beta p_x m c + \beta \alpha_x p_x mc \right.\\
			&\left.\vphantom{\alpha_x^2 p_x^2} + \{\alpha_y, \beta\} p_y mc + \{\alpha_z \beta \} p_z mc 
			+ \{\alpha_x, \alpha_y\} p_x p_y + \{\alpha_y, \alpha_z\} p_y p_z
			+\{\alpha_z, \alpha_x\}p_z p_x
 		\right]
	\end{align*}
Koeffizientenvergleich:
	\begin{align*}
		\alpha_x^2 &= \alpha_y^2 = \alpha_z^2 = 1 = \beta^2
		& &\boxed{\{\alpha_i, \beta \} = 0}\\
		& & &\{\alpha_i, \beta \} = 0 \text{ für } i \neq j \\
		&\Rightarrow \boxed{\{\alpha_i, \beta \} = 2 \delta_{ij}}
	\end{align*}
$\Rightarrow \alpha_i, \beta$ sind keine Zahlen. (vllt Matrizen?)
	\begin{align*}
		\alpha_i &= \alpha_i^\dagger ,&
		\beta &= \beta^\dagger ,& 
		\text{weil } H &= H^\dagger \\
		\text{EW }: \alpha_i^2 &= \mathds{1} ,&
		\beta^2 &= \mathds{1} 
		&\Rightarrow \mathrm{ EW } &\in \{\pm 1\}  
	\end{align*}
	\begin{align*}
		\mathrm{Sp}(\alpha_x) = \mathrm{Sp}(\alpha_x\alpha_y^2) 
		= \mathrm{Sp} (\alpha_y \alpha_x \alpha_y) 
		= - \mathrm{Sp}(\alpha_x \alpha_y^2) = -\mathrm{Sp}(\alpha_x) = 0
	\end{align*}
$\alpha_i, \beta$ sind spurlose hermitesche Matrizen mit EW $\pm 1$
	\begin{align*}
		\text{spurlos } \Rightarrow \# (\mathrm{EW } = + 1)
		= \# (\mathrm{EW } = - 1) \Rightarrow \text{ Dimension }d\text{ gerade}
	\end{align*}
Bei $d=2$ haben wir die Paulimatrizen: Basis über $\mathds{R}$ aller spurlosen hermiteschen $2 \times 2$ Matrizen.
	\begin{align*}
		\sigma_x &=
		\begin{pmatrix}
		0 & 1 \\
		1 & 0
		\end{pmatrix}&
		\sigma_y &=
		\begin{pmatrix}
		0 & -i \\
		i & 0
		\end{pmatrix}&
		\sigma_z &= 
		\begin{pmatrix}
		1 & 0 \\
		0 & -1
		\end{pmatrix} \\
		\{\sigma_i, \sigma_j\} &= 2 \delta_{ij} 
		&\text{Nehme } \sigma_0 &=
		\begin{pmatrix}
		1 & 0 \\
		0 & 1
		\end{pmatrix}
		= \mathds{1}_2 \text{ hinzu} 
	\end{align*}
Problem: $\mathds{1}_2$ ist nicht spurlos

Kleinste mögliche Dimension ist $d = 4$
\\
Dirac- $\alpha,\beta$ Matrizen:
	\begin{align*}
		\vec{\alpha} &=
		\begin{pmatrix}
		0 & \vec{\sigma} \\
		\vec{\sigma} & 0
		\end{pmatrix} 
		& 4 \times 4 \text{ Matrix Beispiel: }
		\alpha_z &=
		\begin{pmatrix}
		0 & \sigma_z \\
		\sigma_z & 0
		\end{pmatrix}
		=
		\begin{pmatrix}
		0 & 0 & 1 & 0 \\
		0 & 0 & 0 & -1 \\
		1 & 0 & 0 & 0\\
		0 & -1 & 0 & 0
		\end{pmatrix} \\
		\beta &=
		\begin{pmatrix}
		\mathds{1}_2 & 0 \\
		0 & \mathds{1}_2
		\end{pmatrix}
	\end{align*}
$\vec{\alpha}, \beta$ sind nicht eindeutig festgelegt, da auch $\vec{\alpha}', \beta'$ mit 
$\alpha_i' = \U^\dagger \alpha_i \U,~ \beta' = \U^\dagger \beta \U$ ($\U$ unitäre Matrix) obige Bedingungen erfüllen.
	\begin{align*}
		-i\hbar \frac{\partial}{\partial t} \ket{\psi}
		&= \left(\vec{\alpha} c \vec{p} + \beta m c^2\right) \ket{\psi} \\
		\text{Ortsraum } i\hbar \frac{\partial \psi (\vec{r}, t)}{\partial t}
		&= \left(-i\hbar c \vec{\alpha} \vec{\nabla} + \beta m c^2\right) \psi (\vec{r}, t)
	\end{align*}
DGL 1.Ordnung in $\frac{\partial}{\partial t}$ und $\frac{\partial}{\partial x}, \frac{\partial}{\partial y}, \frac{\partial}{\partial z}$ aber $\psi (\vec{r}, t)$ hat 4 Komponenten.

$\Rightarrow$ Lorentz-Spinor
\\
Wdh: \marginpar{30.11.2015}