\section{Relativistische Quantenmechanik}
\subsection{Die Dirac-Gleichung}
NR QM:
	\begin{align*}
		H &= \frac{p^2}{2m} 
		& \vec{p} &\mapsto \hat{\vec{p}} = -i \hbar \vec{\nabla} \\
		& & E &\mapsto H = i \hbar \frac{\partial}{\partial t} \\
		i \hbar \frac{\partial}{\partial t} \ket{\psi} 
		&= \frac{\hat{p}^2}{2m} \ket{\psi} 
	\end{align*}
Spezielle Relativitätstheorie:
	\begin{align*}
		H^2 &= p^2 c^2 + m^2 c^4 
		&\Rightarrow H &= \sqrt{p^2c^2 +m^2 c^4} 
		= mc^2\left(1 + \frac{\vec{p}^2}{2m} - \frac{(\vec{p}^2)^2}{8 m^4 c^4} + \ldots\right)
	\end{align*}
Relativistische Schrödinger Gleichung (?):
	\begin{align*}
		i \hbar \frac{\partial}{\partial t} \ket{\psi} 
		= \sqrt{p^2c^2 +m^2 c^4} \ket{\psi}
	\end{align*}
Ortsraum:
	\begin{align*}
		i \hbar \frac{\partial}{\partial t} \ket{\psi} 
		= mc^2 \left(1 - \frac{\vec{\hbar^2}}{2mc^2}\vec{\nabla}^2 - \frac{\hbar^2(\vec{\nabla}^2)^2}{8 m^4 c^4}\right)
	\end{align*}
	\begin{enumerate}
		\item unsymmetrisch im Raum und Zeit (nicht forminvariant unter Lorentz-Transformation)
		\item Hohe Ableitungen
	\end{enumerate}