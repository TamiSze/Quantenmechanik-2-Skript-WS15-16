	wobei \marginpar{15.10.2015}
		\begin{align*}
			\text{Vernichtnungsoperator:~} a &= \frac{1}{\sqrt{2}} 
			\left( y + i \tilde{p} \right) \\
			\text{Erzeugungsoperator:~} a^\dagger &= \frac{1}{\sqrt{2}} 
			\left( y - i \tilde{p} \right)
		\end{align*}
	Und da dann
		\begin{align*}
			[a , a^\dagger] &= 1 &\text{~oder~} a a^\dagger &= a^\dagger a + 1 \\
			\Rightarrow \tilde{H} = \left( a^\dagger a + \frac{1}{2} \right)& \\
			\text{und somit~} 
			y &= \frac{1}{\sqrt{2}} \left( a + a^\dagger \right),& 
			\tilde{p} &= \frac{1}{i\sqrt{2}} \left( a - a^\dagger \right)
		\end{align*}
	Sei nun:
		\begin{align*}
			\tilde{H} \ket{n} &= \epsilon_n \ket{n} &\left( H \ket{n} = E_n \ket{n} 
			\Rightarrow E_n = \hbar \omega \epsilon_n \right)
			a^\dagger a \ket{n} = \left( \epsilon_n - \frac{1}{2} \right) \ket{n}
		\end{align*}
	(n ist Anzahl der Knoten (Nullstellen) der Wellenfunktionen.)
	Aus QM 1:
		\begin{align*}
			a \ket{n} &= \sqrt{n} \ket{n-1} \\
			a^\dagger \ket {n} &= \sqrt{n+1} \ket{n+1} \\
			&\Rightarrow \epsilon_{n+1} = \epsilon_n + 1
		\end{align*}
	mit $n = 0, 1, 2 \ldots$: \footnote{\text{N: Nummernoperator (Teilchenzahloperator)}}
		\begin{align*}
			&\epsilon_n = n + \frac{1}{2} \leadsto E_n = \hbar \omega \left(n + \frac{1}{2} \right) \\
			&\text{und~} \underbrace{a^\dagger a}_{\substack{N}}
			%\footnote{\text{Nummernoperator (Teilchenzahloperator)}}
			\ket{n} = n \ket{n}
		\end{align*} 
	Somit ist $\tilde{H} = N + \frac{1}{2}$

\section{Zeitabhängige Störung}
%\numberwithin{equation}{chapter}
%\setcounter{equation}{1}
	\subsection{Zeitabhängige Störungstheorie und das Wechselwirkungsbild}
	Betrachte 
		\begin{equation*}
			H(t) = H^0 + H^1 (t)
		\end{equation*}
	$H^0$ ist zeitunabhängig: $H^1(t) = 0$ für $t < t_i$ oder $t > t_f$
	Ungestörtes System: 
		\begin{align*}
			H^0 \ket{n} &= E_n \ket{n} = \hbar \omega_n \ket{n} \\
			\ket{\psi(t)} &= \sum_n \braket{n | \psi(t_0)} e^{-i \omega_n (t-t_0)} \ket{n} \text{~für~} t, t_0 < t_i (\text{oder~} t, t_0 > t_f)
		\end{align*}
	Von nun an: $\ket{\psi(t_0)} = \ket{n}$ für $t_0 < t_i$
		\begin{equation*}
			\Rightarrow \boxed{\ket{\psi (t)} = e^{-i \omega_n (t-t_0)} \ket{n}
			} \text{~für~} t_0, t < t_i
		\end{equation*}		
	Formale Lösung von 
		\begin{equation*}
			i \hbar \frac{\diff}{\diff t} \ket{\psi(t)}
			= H^0 \ket{\psi(t)}
		\end{equation*}			
	Ist
		\begin{equation*}
			\Rightarrow \ket{\psi(t)} = \text{T} e^{-\frac{i}{\hbar} H^0 (t-t_0)} 
			\ket{\psi(t_0)} = \U^0(t-t_0) \ket{\psi(t_0)}
		\end{equation*}
	Achntung: T ist Zeit\underline{ordnungs}operator und $\U^0(t-t_0)$ ist Zeit\underline{entwicklungs}operator.
	Eigenschaften:
		\begin{align*}
			\left.
			\begin{aligned}
				\U^0(t_1) \U^0(t_2) &= \U^0(t_1 + t_2) \\
				\U^0(-t) &= \U^{0\dagger} (t)		
			\end{aligned}
			\right\} \U^{0\dagger} (t) \U^0 (t) = \mathds{1}
		\end{align*}
		\begin{align*}
			1 &= \braket{\psi(t) | \psi(t)} \\
			&= \braket{\psi(t_0) | \U^{0 \dagger} (t-t_0) \U^0 (t-t_0) | \psi(t_0)} \\
			&= \braket{\psi(t_0) | \psi(t_0)}
		\end{align*}
	Norm bleibt erhalten $\Rightarrow U^0$ ist unitär! \\
	Betrachte nun 
		\begin{equation*}
			i \hbar \frac{\diff}{\diff t} \ket{\psi(t)} = H(t) \ket{\psi(t)}
		\end{equation*}
	Endzustand ($t>t_f$):
		\begin{equation*}
			\boxed{ \ket{\psi(t)} = 
			\sum_m c_m(t) e^{-i \omega_m (t-t_0)} \ket{m}
			}
		\end{equation*}
	Übergangswahrscheinlichkeit von $\ket{n}$ in den Zustand $\ket{m}$:
		\begin{equation*}
			\boxed{P_{mn} (t) = |c_m(t)|^2}
		\end{equation*}
	Das gestörte System
		\begin{equation*}
			i \hbar \frac{\diff}{\diff t} \ket{\psi(t)}
			= \left( H^0 + H^1(t) \right) \ket{\psi(t)}
		\end{equation*}
	wird im Allgemeinen nicht durch
		\begin{equation*}
			\ket{\psi(t)} = \text{T} e^{-\frac{i}{\hbar} H(t) (t-t_0)} \ket{\psi(t_0}
		\end{equation*}
	gestört! $H(t)$ ist zeitabhängig.
	Aber 
		\begin{align*}
			\braket{\psi(t) | \psi(t)} &= \braket{\psi(t_0) | \psi(t_0)} \\
			\Rightarrow \ket{\psi(t)} &= \U(t-t_0) \ket{\psi(t_0)}
		\end{align*}
	$\U(t,t_0)$ ist ein noch unbekannter unitärer Zeitentwicklungsoperator.\\
	$\ket{\psi(t)}$ ist Wellenfunktion im Schrödingerbild.
	
	Betrachte:
		\begin{equation*}
			\ket{\psi_W(t)} = {(\U^0)}^{-1} (t-t_0) \ket{\psi(t)} 
			= \text{T} e^{\frac{i}{\hbar} H^0 (t-t_0)} \ket{\psi(t)} 
		\end{equation*}
	Für $H^1=0$ gilt:\footnote{hier ist wieder das Runtergeschriebene W für Wechselwirkung und H für Heisenberg.}
		\begin{equation*}
			\ket{\psi_W(t)} = \ket{\psi(t_0)} = \ket{\psi_W} = \ket{\psi}_H
		\end{equation*}
	Nun ist aber $H^1(t) \neq 0$
		\begin{align}
			i\hbar \frac{\diff}{\diff t} \ket{\psi_W(t)} &=
			i\hbar \frac{\diff}{\diff t} \left[\text{T} e^{\frac{i}{\hbar} H^0(t-t_0)} \ket{\psi(t)}\right] \nonumber \\ \nonumber
			&= \text{T} e^{\frac{i}{\hbar} H^0(t-t_0)} [ -H^0 + \underbrace{i\hbar \frac{\diff}{\diff t}] \ket{\psi(t)}}_{\mathclap{N \text{: Nummernoperator(Teilchenzahloperator)}}} \\ \nonumber
			&= \text{T} \left\{ e^{\frac{i}{\hbar} H^0(t-t_0)} H^1(t) e^{-\frac{i}{\hbar} H^0(t-t_0)} \ket{\psi_W(t)}\right\}  \\ 
			\boxed{
		\begin{aligned}i\hbar \frac{\diff}{\diff t} \ket{\psi_W(t)} 
				&= H^1_W(t) \ket{\psi_W(t)}  \\ 
				H^1_W(t)& = \U^{0\dagger} (t-t_0) H^1(t) \U^0(t-t_0) 
		\end{aligned}
			} \label{Wechselwirkung}
		\end{align}
		\begin{tabular*}{\linewidth}{l l l}
			Bild & Zustand & Operator \\
			Heisenberg & zeitunabhängig & Entwicklung durch $H$ \\
			Schrödinger & Zeitentwicklung durch $H$ & Zeitunabhängig (hier aber $H(t))$ \\
			Wechselwirkungs- & Entwicklung durch $H^1(t)$ & Entwicklung durch $H^0$
		\end{tabular*} %bitte noch abtrennungen einführen
		 \\ \\
	Berechnung der $c_m(t)$:
	
	Aus dem Schrödingerbild: 
		\begin{align}
			\ket{\psi(t)} &= \sum_m c_m(t) \ket{m} e^{-i \omega_m (t-t_0)} ,& t&>t_f \nonumber \\
			\ket{\psi_W(t)} &= \sum_m c_m(t) \ket{m} &\Rightarrow c_m(t) &= \braket{m | \psi_W(t)} \nonumber \\
			i\hbar \frac{\diff}{\diff t} c_m(t) 
			&= \Braket{m | i\hbar \frac{\diff}{\diff t} | \psi(t)} \nonumber\\
			&\overset{\eqref{Wechselwirkung}}{=} \braket{m | H^1_W(t) | \psi_W(t)} \nonumber\\
			&= \sum_k \braket{m | H^1(t) | k} \underbrace{\braket{k | \psi_W(t)}}_{\substack{c_k(t)}} \nonumber\\ 
			\label{gekopp-nlinear-Sys}
			&= \sum_k \braket{m | H^1(t) | k} e^{-i(\omega_k - \omega_m)(t - t_0)} c_k(t) 
		\end{align}
	Anfangsbedingung: $(t_0 < t_i \Rightarrow H^1(t_0) = 0):$
		\begin{align*} 
			c_k(t_0) = \braket{k | \psi_n(t_0)} = \braket{k | \psi(t_0)} = \braket{k | n} = \delta_{kn}
		\end{align*} 	
	Das gekoppelte nicht-lineare System von Differentialgleichungen \eqref{gekopp-nlinear-Sys} ist nicht leicht lösbar. $\Rightarrow$ Betrachte \glqq kleine\grqq{} Störungen $H^1(t)$: 
		\begin{align*}
			\ket{\psi_W(t)} &= \underbrace{\ket{\psi(t_0)}}_{\substack{=\ket{\psi_W(t_0)}}} 
			\frac{i}{\hbar} \int_{t_0}^{t} \diff t' H^1_W(t') \ket{\psi(t_0)} + \ldots 
			\left( \text{linearisierung von~} i\hbar \frac{\diff}{\diff t} \ket{\psi_W(t)} = H^1(t) \ket{\psi_W(t)}\right) \\
			c_m(t) &= \braket{m | \psi_W(t)} 
			= \delta_{mn} - \frac{i}{\hbar} \sum_k \int_{t_0}^{t} \diff t' \braket{m | H^1(t') | k} e^{-i(\omega_n - \omega_m)(t'-t_0)} 
			\underbrace{c_k(t_0)}_{\substack{\delta_{kn}}} + \ldots
		\end{align*}
	Setze $t_0=0 (t_i>0)$
		\begin{equation*}
			\Rightarrow \boxed{c_m(t) = \delta_{mn} 
				- \frac{i}{\hbar} \int_{t_0}^{t} \diff t' \braket{m | H^1(t') | n} e^{-i(\omega_n - \omega_m) t'} + \ldots}
		\end{equation*}
	Formale Lösung zu allen Ordnungen in $H^1$:	
		\begin{align*}
			\ket{\psi_W(t)} &= \U^{0\dagger} (t-t_0) \ket{\psi(t)}
			= \U^{0\dagger} (t-t_0) \underbrace{\U (t,t_0)}_{\substack{t>t_f}} \ket{\psi_W(t)}
		\end{align*}
		\begin{align*}
		\boxed{	
		\begin{aligned}
			\ket{\psi_W(t)} &= W(t,t_0) \ket{\psi(t_0)} &\text{mit~} W(t,t_0) &= \U^{0\dagger} (t-t_0) \U (t,t_0) 
		\end{aligned}
		}
		\end{align*}
		\begin{equation*}
			i \hbar \frac{\diff}{\diff t} \ket{\psi_W(t)} = H^1_W(t) \ket{\psi_(t)}
		\end{equation*}	
		\begin{align*}
		\Rightarrow
		\boxed{	
			\begin{aligned}
				i \hbar \frac{\diff}{\diff t} W(t,t_0) 
				&= H^1_W(t) W(t,t_0) &\text{für~} t>t_0 \text{~mit~} W(t_0,t_0) = \mathds{1}
			\end{aligned}
		}
		\end{align*}
		\begin{equation*}
			\boxed{ W(t,t_0) = \mathds{1} 
				- \frac{i}{\hbar} \int_{t_0}^{t} \diff t' H^1_W(t') W(t',t_0)
				}
		\end{equation*}
	Integralgleichung: Lösung durch Iteration (Dyson-Reihe)
		\begin{align*}
			W^0(t,t_0) &= \mathds{1} \\
			W^n(t,t_0) &= \mathds{1} - \frac{i}{\hbar} \int_{t_0}^{t} \diff t' H^1_W(t') W^{n-1}(t',t_0)
		\end{align*}
	\underline{Beispiele:}
		\begin{align*}
			W^1(t,t_0) &= \mathds{1}
			- \frac{i}{\hbar} \int_{t_0}^{t} \diff t' H^1_W(t') \underbrace{W^{0}(t',t_0)}_{\substack{\mathds{1}}} \\
			W^2(t,t_0) &= \mathds{1} 
			- \frac{i}{\hbar} \int_{t_0}^{t} \diff t' H^1_W(t')
			+ \left( -\frac{i}{\hbar}\right)^2 \int_{t_0}^{t} \diff t' \int_{t_0}^{t'} \diff t'' H^1_W(t') H^1_W(t'') \\
			W^n(t,t_0) &= \mathds{1} 
			+ \sum_{n=1}^{\infty} \left(-\frac{i}{\hbar}\right)^n 
			\int_{t_0}^{t} \diff t_n
			\int_{t_0}^{t_n} \diff t_{n-1} \cdots 
			\int_{t_0}^{t_2} \diff t_1 H^1_W(t_n) H^1_W(t_{n-1}) \cdots H^1_W(t_1)
		\end{align*}