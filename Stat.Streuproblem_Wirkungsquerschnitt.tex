\section{Streutheorie}
	\subsection{Stationäres Streuproblem und Wirkungsquerschnitt}
	BILD
		
	Stromdichte des einlaufenden Teilchenstrahls: $\vec{j}_{ein}$
	\\
	BILD
	
	Anzahl der Teilchen, die durch $\diff \vec{A}$ läuft:
		\begin{align*}
			\diff N &= 
			\underbrace{\vec{j}_{ein} \cdot \diff \vec{A}}_{\mathclap{\text{Strom}}}
			\cdot \diff t
		\end{align*}
	NOCH EIN BILD
		\begin{align*}
			\diff A &= r^2 \diff \Omega = r^2 \diff \Theta \sin \Theta \diff \phi
		\end{align*}
	Differentieller Wirklungsquerschnitt $(r \rightarrow \infty)$
		\begin{empheq}[box=\boxed]{align*}
			\frac{\diff \sigma}{\diff \Omega} &=
			\frac{1}{|\vec{j}_{ein}|} \frac{\diff N}{\diff \Omega ~\diff t}
		\end{empheq}
	Homogener Strahl von Punktteilchen : $\vec{j} = \vec{v} \rho$ ($\rho$ ist Dichte)
		\begin{empheq}[box=\boxed]{align*}
			\frac{\diff \sigma}{\diff \Omega} &=
			\frac{|\vec{j}_{aus}(r, \Theta, \phi)|}{|\vec{j}_{ein}|} ~r^2
		\end{empheq}
	$ \leadsto |\vec{j}_{aus}| \propto \frac{1}{r^2}$ (bei gleichen Detektorflächen $\diff A$)
	
	Totaler Wirkungsquerschnitt:
		\begin{align*}
			\sigma = \int \diff \Omega \frac{\diff \sigma}{\diff \Omega}
		\end{align*}
	Beispiel: Streuung an Scheibe der Fläche A:
	\\
	BILD
	\\
	