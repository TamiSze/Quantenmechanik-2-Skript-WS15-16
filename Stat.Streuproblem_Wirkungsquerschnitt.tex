\section{Streutheorie}
	\subsection{Stationäres Streuproblem und Wirkungsquerschnitt}
	BILD
		
	Stromdichte des einlaufenden Teilchenstrahls: $\vec{j}_{ein}$
	\\
	BILD
	
	Anzahl der Teilchen, die durch $\diff \vec{A}$ läuft:
		\begin{align*}
			\diff N &= 
			\underbrace{\vec{j}_{ein} \cdot \diff \vec{A}}_{\mathclap{\text{Strom}}}
			\cdot \diff t
		\end{align*}
	NOCH EIN BILD
		\begin{align*}
			\diff A &= r^2 \diff \Omega = r^2 \diff \Theta \sin \Theta \diff \phi
		\end{align*}
	Differentieller Wirklungsquerschnitt $(r \rightarrow \infty)$
		\begin{empheq}[box=\boxed]{align*}
			\frac{\diff \sigma}{\diff \Omega} &=
			\frac{1}{|\vec{j}_{ein}|} \frac{\diff N}{\diff \Omega ~\diff t}
		\end{empheq}
	Homogener Strahl von Punktteilchen : $\vec{j} = \vec{v} \rho$ ($\rho$ ist Dichte)
		\begin{empheq}[box=\boxed]{align*}
			\frac{\diff \sigma}{\diff \Omega} &=
			\frac{|\vec{j}_{aus}(r, \Theta, \phi)|}{|\vec{j}_{ein}|} ~r^2
		\end{empheq}
	$ \leadsto |\vec{j}_{aus}| \propto \frac{1}{r^2}$ (bei gleichen Detektorflächen $\diff A$)
	
	Totaler Wirkungsquerschnitt:
		\begin{align*}
			\sigma = \int \diff \Omega \frac{\diff \sigma}{\diff \Omega}
		\end{align*}
	Beispiel: Streuung an Scheibe der Fläche A:
	\\
	BILD
	\\
		\begin{align*}
			\sigma &= \int \diff \Omega \frac{\diff \sigma}{\diff \Omega}
			= \frac{1}{|\vec{j}_{ein}|} \int \diff \Omega \frac{\diff N}{\diff \Omega ~\diff t}\\
			&= \frac{1}{|j_{ein}|} \frac{\diff N}{\diff t} & &\frac{\diff N}{\diff t}: 
			\text{~Zahl gestreuter Teichen pro Zeit} \\
			&= \frac{1}{|j_ein|} \cdot |j_{ein}| \cdot A = A
		\end{align*}
	Schrödingergleichung: 
		\begin{equation*}
			i \hbar \frac{\partial \Psi (\vec{r}, t)}{\partial t} = H \Psi (\vec{r} , t) 
		\end{equation*}
	Einlaufendes Wellenpaket
	
	BILD
	
		\begin{equation*}
			\psi (\vec{r} , t) = 
			\int \frac{\diff^3 k}{(2 \pi)^3} A(\vec{k}) e^{i(\vec{k} \vec{r} - \omega(\vec{k}) t)}
		\end{equation*}
	Annahme:
		\begin{align*}
			v_G << c &\Rightarrow \omega(\vec{k}) = \frac{\hbar \vec{k}^2}{2 m} & &(V(\vec{r}) = 0 (\text{Potential}))
		\end{align*}
	Nebenbemerkung:
	
	Wir kennen bislang nur die nichtrelativistische Quantenmechanik. Die Streutheorie relativistischer Teilchen unterscheidet sich nicht \underline{wesentlich}.
	
	Bild
	
		\begin{align*}
			r &>> R ,& r &>> \lambda = \frac{2 \pi}{k} &\Rightarrow 
			&\text{gestreute Welle ist \underline{lokal} eine Kugelwelle} 
		\end{align*}
	Betrachte \underline{elastische} Streuung an Kugelsymmetrischem Störpotenial $V(r)$:
		\begin{align*}
			\psi_S (\vec{r}, t) &= 
			\int \frac{\diff^3 k}{(2 \pi)^3} A(\vec{k}) \phi_S (\vec{r}, \vec{k})
			e^{-i \omega (\vec{k}) t} \\
			&\text{Einfallender Strahl:} & \phi_0 (\vec{r}) &= e^{i \vec{k}_0 \vec{r}} = e^{i k_0 z}\\
			&\text{Gestreuter Strahl:} & \phi_S (\vec{r}) 
		\end{align*}
	Für $r >> R$: Asymptotisch freie Wellen
	
	Einfallender Strahl:
		\begin{align*}
			\vec{p} &= \hbar k_0 \vec{e}_z ,& E &= \frac{\hbar^2 k_0}{2 m}\\
			-\frac{\hbar^2}{2 m} \vec{\nabla}^2 \phi_0 (\vec{r}) &= E \phi_0 (\vec{r})
		\end{align*}
	Wahrscheinlichkeitsstromdichte
		\begin{align*}
			\vec{j}_0 (\vec{r}, t) &=
			\frac{1}{2 m} 
			\left(\psi^* (\vec{r} , t) \vec{p} \psi (\vec{r}, t) - \psi (\vec{r}, t) \vec{p} \psi^* (\vec{r}, t)
			\right) \\
			&= \frac{\hbar k_0}{m} \vec{e}_z
		\end{align*}
	Nebenbemerkung: Teilchenstrom für $N_0$ Teilchen: $N_0 \frac{\hbar k}{m} \vec{e}_z$
	
	zu lösen ist:
		\begin{empheq}[box=\boxed]{align*}
			H \phi(\vec{r}) &= E \phi (\vec{r}) & &\text{für } E= \frac{\hbar k_0^2}{2 m} > 0\\
			H &= \frac{\vec{p}^2}{2 m} + V (r) & &V(r) \text{ begrenzt auf } r<R \\
			\phi (\vec{r}) &= e^{i k z} + \phi_S (\vec{r})
		\end{empheq}
	Ansatz:
		\begin{align*}
			\phi_S (\vec{r}) &\underset{r \rightarrow \infty}{\sim} f(\Theta) \frac{e^{i k_0 r}}{r} &
			&\text{mit Streuamplitude } f(\Theta)\\
			\vec{j} &= \frac{\hbar k_0}{m} \frac{|f(\Theta)|^2}{r^2}
			\hat{e}_r (\Theta , \phi) + \mathscr{O} \left(\frac{1}{r^3}\right)
		\end{align*}
		\begin{empheq}[box = \boxed]{align*}
			\frac{\diff \sigma}{\diff \Omega} = r^2 \frac{|\vec{j}_S|}{|\vec{j}_0|}
			= |f(\Theta)|^2
		\end{empheq}
	Dieser Ansatz löst die Schrödingergleichung für $r \rightarrow \infty$ 
		\begin{align*}
			\underset{r \rightarrow \infty}{\lim} r V(r) &= 0 &
			&(\text{kein Coulomb-Potential})
		\end{align*}
	Streutheorie \marginpar{5.11.15}
	
	Kurze Wiederholung \marginpar{heute nicht Bali}
	
	BILD
	
		\begin{align*}
			\diff A &= r^2 \diff \Omega ,& 
			\vec{j} &= \frac{\hbar}{2 m i} \left(\psi^* \vec{\nabla} \psi - \psi \vec{\nabla} \psi^* \right)
		\end{align*}
	$\diff N$ in diesem $\diff A$, in einem kleinen Zeitraum $\diff t$ \marginpar{??}
		\begin{align*}
			\frac{\diff \sigma}{\diff \Omega}
			&= \frac{|\vec{j}_{Streu}| ~\diff t ~ \diff A}{ |\vec{j}_{ein}| ~\diff t ~ \diff \Omega}
			=\frac{|\vec{j}_{Streu}| r^2}{|\vec{j}_{ein}|} :&
			&\text{hat Dimension Fläche}
		\end{align*}
	Annahmen:
		\begin{itemize}
			\item[-] 	Elastische Streuung, $\hbar k = \hbar k'$
			\item[-]	Stationäre Situation
			\item[-]	$V(\vec{r}) = V(|\vec{r}|)$ (Kugelsymmetrie)
			\item[-]	keine Absorption oder Emission
		\end{itemize}
	Wir wissen
		\begin{align*}
			\phi = \phi_0 + \phi_S = e^{i k z} + \phi_S
		\end{align*}
	$\phi$ ist im Bereich $r>R$ übereinstimmend mit Lösung der freien Schrödinger Gleichung.
		\begin{align*}
			\phi = \sum_{\ell = 0}^{\infty} R_\ell (r) 
			\underbrace{P_\ell (\cos \Theta)}_{\mathclap{\sqrt{\frac{4 \pi}{2\ell+1}} 
					Y_{\ell 0}(\cos \Theta)}} 
			& &[H, L_z] = 0 ~(\text{Drehimpulserhaltung})
		\end{align*}
	BILD \\
	\underline{Legendre-Polynome $P_\ell (\cos \Theta)$}
		\begin{align*}
			P_0 (\cos \Theta) &= P_0 (z) = 1\\
			P_1 (z) &= z \\
			P_2 (z) &= \frac{1}{2} (3z^2-1) \\
			(\ell + 1) P_{\ell + 1} (z) &= (2 \ell + 1) z P_\ell (z) - \ell P_{\ell - 1} (z)
		\end{align*}